\documentclass[12pt]{article}
\usepackage{graphicx}
\usepackage{wrapfig}
\usepackage{tikz}
\usepackage{ucs} 
\usepackage[T1,T2A]{fontenc}
\usepackage[utf8x]{inputenc} % Включаем поддержку UTF8  
\usepackage[russian]{babel}  % Включаем пакет для поддержки русского языка  
\usepackage{amsfonts}
\usepackage[left=1.5cm,right=1.5cm]{geometry}
\usepackage{amsmath}
\usepackage{amssymb}

\usetikzlibrary{decorations.markings}
\usetikzlibrary{patterns}
\usetikzlibrary{calc}
\usetikzlibrary{arrows}
\title{Задачи к 6 лекции}  
\author{Нечитаев Дмитрий}

\begin{document} 
	\maketitle
	\subsection*{Упражнение 1}
	Для уравнения
	\[\ddot{x} + x = -\epsilon x^3 \eqno(1)\]
	Нужно найти поправку второго порядка в рамках теории возмущений по малому параметру $\epsilon$ для начальных условий $x(0)=a,\;\dot{x}(0)=0$.
	
	
	Будем искать решение в виде: $x =x_0+ \epsilon x_1 + \epsilon^2 x_2$, тогда подстановка в уравнение (1) и группирование членов с одинаковыми подяками дает нам систему:
	\[\begin{cases}
	\ddot{x_0}+x_0 = 0\\
	\ddot{x_1}+x_1 = -x_0^3 \\
	\ddot{x_2} +x_2  = -3x_0^2x_1
	\end{cases}\]
	Общее решение уравнения $\ddot{x}(t)+x(t) = -y(t)$ записывается в виде:
	\[x(t) = \Big[x(0)+\int_0^{t}y(\tau)\sin\tau d\tau\Big]\cos t - \sin t \Big[\dot{x}(0)+\int_{0}^{t}y(\tau)\cos\tau d\tau\Big] \eqno(2) \] 
	Положим что $x_0(0) = a,\; x_1(0) = x_2(0) = 0,\; \dot{x_0}(0) = \dot{x_1}(0) = \dot{x_2}(0) = 0$, тогда выражения для поправок примут вид\footnote{Второе уравниение в системе было получено на лекции}:
	\[\begin{cases}
	x_0(t) = a\cos(t) \\
	x_1(t) = -\frac{a^3}{16}\Big(6t\sin t +\frac{1}{2}\cos t - \frac{1}{2}\cos 3t\Big)
	\end{cases} \eqno(3)\]
	Вычисляем $x_0^2 x_1$:
	\[x_0^2x_1 = -a^2\cos^2 t \frac{a^3}{16}\Big(6t \sin t + \frac{1}{2}\cos t - \frac{1}{2} \cos 3t\Big) = -\frac{a^5}{32}(1+\cos2t)\Big(6t \sin t + \frac{1}{2}\cos t - \frac{1}{2} \cos 3t\Big) = \]
	\[ = -\frac{a^5}{32}\Big(6t\sin t + \frac{1}{2}\cos t - \frac{1}{2}\cos3t+6t\sin t\cos 2t+\frac{1}{2}\cos t \cos 2t -\frac{1}{2}\cos3t \cos 2t\Big) = \]
	\[ = -\frac{a^5}{32}\Big(3t\sin t + 3t \sin 3t + \frac{1}{2}\cos t - \frac{1}{4}\cos 3t - \frac{1}{4}\cos 5t\Big) \eqno(4)\]
	Тогда $-3x_0^2x_1$:
	\[-3x_0^2x_1 = \frac{3a^5}{32}\Big(3t\sin t + 3t \sin 3t + \frac{1}{2}\cos t - \frac{1}{4}\cos 3t - \frac{1}{4}\cos 5t\Big) \eqno(5)\]
	Вычислим пару интегралов:
	\[I_1 = \int_0^\tau (-3x_0^2x_1) \sin t dt = \frac{3a^5}{32}\int_0^\tau dt \Big(3t\sin t + 3t \sin 3t + \frac{1}{2}\cos t - \frac{1}{4}\cos 3t - \frac{1}{4}\cos 5t\Big)\sin t = \]
	\[\frac{3a^5}{32}\int_0^\tau dt \Bigg(3t\Big(\frac{1-\cos 2t}{2} + \frac{\cos 2t - \cos 4t}{2}\Big) + \frac{\sin 2t}{4} + \frac{\sin 2t -\sin 4t}{8} + \frac{\sin 4t -\sin 6t}{8}\Bigg) = \]
	\[ \frac{3a^5}{32}\int_0^\tau dt \Bigg(3t\cdot\frac{1 - \cos 4t}{2} + \frac{\sin 2t}{4} + \frac{\sin 2t -\sin 6t}{8}\Bigg) = \frac{3a^5}{32\cdot96}\Big(72t^2-36t\sin 4t - 18 \cos 2t - 9\cos 4t + 2 \cos 6t\Big)\Bigg | _ 0 ^ \tau = \]
	\[\frac{a^5}{1024}\big(72\tau^2-36\tau\sin 4\tau - 18 \cos 2\tau - 9\cos 4\tau + 2 \cos 6\tau + 25\big) \eqno(6)\]
	\[I_2 = \int_0^\tau (-3x_0^2x_1) \cos t dt = \frac{3a^5}{32}\int_0^\tau dt \Big(3t\sin t + 3t \sin 3t + \frac{1}{2}\cos t - \frac{1}{4}\cos 3t - \frac{1}{4}\cos 5t\Big)\cos t = \]
	\[\frac{a^5}{1024}\big(24\tau+78\sin 2\tau + 3\sin 4\tau - 2\sin 6\tau - 144\tau\cos 2\tau - 36\tau\cos 4\tau\big) \eqno(7)\]
	Выражение для второй поправки:
	\[x_2(t) = \frac{a^5\sin t}{1024}\Big(24t+78\sin 2t + 3\sin 4t - 2\sin 6t - 144t\cos 2t - 36t\cos 4t\Big) - \]
	\[- \frac{a^5\cos t}{1024}\big(72t^2-36t\sin 4t - 18 \cos 2t - 9\cos 4t + 2 \cos 6t + 25\big) =\]
	\[= \frac{a^5}{1024}\Big(-72t^2\cos t+96t\sin t-36t\sin 3t + 23 \cos t -24\cos3t+\cos5t\Big)\eqno(8)\]
	Данный ответ работает при тех временах, когда вклад от второй поправки много меньше чем от первой, т.е. время до которого приближение работает хорошо определяется соотношением:
	\[\frac{a^3}{16}\cdot 6t = \epsilon \cdot \frac{72 a^5t^2}{1024} \Rightarrow t = \frac{3}{8}\cdot\frac{1024}{72}\frac{1}{a^2\epsilon} \Rightarrow t \sim \frac{1}{a^2\epsilon}\]
	
\end{document}
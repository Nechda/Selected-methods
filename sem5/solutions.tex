\documentclass[12pt]{article}
\usepackage{graphicx}
\usepackage{wrapfig}
\usepackage{tikz}
\usepackage{ucs} 
\usepackage[T1,T2A]{fontenc}
\usepackage[utf8x]{inputenc} % Включаем поддержку UTF8  
\usepackage[russian]{babel}  % Включаем пакет для поддержки русского языка  
\usepackage{amsfonts}
\usepackage[left=1.5cm,right=1.5cm]{geometry}
\usepackage{amsmath}
\usepackage{amssymb}

\usetikzlibrary{decorations.markings}
\usetikzlibrary{patterns}
\usetikzlibrary{calc}
\usetikzlibrary{arrows}
\title{Задачи к 5 лекции}  
\author{Нечитаев Дмитрий}

\begin{document} 
	\maketitle
	\section*{Упражнение 1}
	\subsection*{Лагранжиан в ДСК}
	Выражение для Лагранжиана свободной частицы в интегрциальной системе отсчета с ДПСК:
	\[L = \frac{mv^2}{2} = \frac{m}{2}\big(\dot{x}^2 + \dot{y}^2 + \dot{z}^2\big) \eqno(1)\]
	Находим уравнение движения:
	\[\frac{d}{dt}\frac{\partial L}{\partial \dot{q_i}} = \frac{\partial L}{\partial q_i} \eqno(2)\]
	\[m\ddot{q_i} = 0 \eqno(3)\]
	\subsection*{Лагранжиан в цилиндрических координатах}
	
	\[\begin{cases}
	x = r \cos \phi \\
	y = r \sin \phi \\
	z = z
	\end{cases} \Rightarrow 
	\begin{cases}
	\dot{x} = \dot{r}\cos \phi - r\sin(\phi) \dot{\phi} \\
	\dot{y} = \dot{r}\sin \phi + r\cos(\phi)\dot{\phi} \\
	\dot{z} = \dot{z}
	\end{cases}
	\eqno(4)
	\]
	После возведения в квадрат и подстановки в уравнение (1), получаем:
	\[L = \frac{m}{2}\big( \dot{r}^2+r^2\dot{\phi}^2 + \dot{z} \big) \eqno(5)\]
	Уравнения движения:
	\[\begin{cases}
	m\ddot{r} = mr\dot{\phi}^2 \\
	\frac{d}{dt}\big(mr^2\dot{\phi}^2\big) = 0\\
	m\ddot{z} = 0
	\end{cases} \eqno(6)\]
	\subsection*{Лагранжиан в сферических координатах}
	\[\begin{cases}
	x = r\cos \theta \cos \phi \\
	y = r \cos \theta \sin \phi\\
	z = r \sin \theta
	\end{cases} \Rightarrow 
	\begin{cases}
	\dot{x} = \dot{r}\cos \theta \cos\phi - r\dot{\theta}\sin\theta\cos\phi  - r\dot{\phi}\cos\theta\sin\phi \\
	\dot{y} = \dot{r}\cos \theta \sin\phi - r\dot{\theta}\sin\theta\sin\phi + r\dot{\phi}\cos\theta\cos\phi \\
	\dot{z} = \dot{z}\sin \theta + r\dot{\theta}\cos \theta 
	\end{cases} \eqno(7)\]
	Тогда Лагранжиан примет вид:
	\[L = \frac{m}{2}\big( \dot{r}^2+r^2\dot{\theta}^2 + r^2\dot{\phi}^2\sin^2\theta  \big) \eqno(8)\]
	Уравнения движения:
	\[\begin{cases}
	m\ddot{r} =  mr\dot{\theta}^2 + mr\dot{\phi}^2\sin^2\theta\\
	\frac{d}{dt}\big(mr^2\dot{\theta}\big) = mr^2\dot{\phi}^2\sin(\theta)\cos(\theta) \\
	\frac{d}{dt}\big(mr^2\sin^2(\theta)\dot{\phi}\big) = 0
	
	\end{cases} \eqno(9)\]
	
	\subsection*{Лагранжиан в кривольниейных координатах}
	Пусть нам теперь дан закон преобразования координат:
	\[\begin{cases}
	x = x(\xi,\eta,\zeta) \\
	y = y(\xi,\eta,\zeta) \\
	z = z(\xi,\eta,\zeta) \\
	\end{cases} \Rightarrow D = \begin{pmatrix}
	\partial x / \partial \xi && \partial x / \partial \eta && \partial x / \partial \zeta \\
	\partial y / \partial \xi && \partial y / \partial \eta && \partial y / \partial \zeta \\
	\partial z / \partial \xi && \partial z / \partial \eta && \partial z / \partial \zeta \\
	\end{pmatrix} \eqno(10)\]
	Тогда справедливо соотношение между дифференциалами:
	\[\begin{pmatrix}
	dx \\ dy \\ dz
	\end{pmatrix} = D \begin{pmatrix}
	d\xi \\ d\eta \\ d\zeta
	\end{pmatrix} \Rightarrow 
	\begin{pmatrix}
	dx & dy & dz 
	\end{pmatrix} = 
	\begin{pmatrix}
	d\xi &  d\eta & d\zeta
	\end{pmatrix} D^T \eqno(11)\]
	Перемножая строчку и столбец можно определить выражение для $dl^2$:
	\[dl^2 = \begin{pmatrix}dx & dy & dz\end{pmatrix} \begin{pmatrix}dx \\ dy \\dz \end{pmatrix} = \begin{pmatrix} d\xi &  d\eta & d\zeta \end{pmatrix} D^T D \begin{pmatrix}
	d\xi \\ d\eta \\ d\zeta
	\end{pmatrix} \eqno(12)\]
	Замечаем, что $v^2dt^2 = dl^2$, т.е Лагранжиан преобретает вид:
	\[L = \frac{m}{2} \begin{pmatrix} \dot{\xi} &  \dot{\eta} & \dot{\zeta} \end{pmatrix} D^T D \begin{pmatrix}\dot{\xi} \\  \dot{\eta} \\ \dot{\zeta} \end{pmatrix} \eqno(13)\]
	Пусть $\{q_1,q_2,q_3\}$ - новые координаты, а $\{x_1,x_2,x_3\}$ - старые координаты, тогда:
	\[D = \Big(\frac{\partial x^i}{\partial q^j}\Big) \Rightarrow L = \frac{m}{2}\Big(\dot{q}^j\frac{\partial x^i}{\partial q^j}\frac{\partial x^i}{\partial q^k} \dot{q}^k \Big) = \frac{m}{2}\Big(\dot{q}^j\dot{q}^k\frac{\partial x^i}{\partial q^j}\frac{\partial x^i}{\partial q^k} \Big) \eqno(14)\]
	Из выражения (14) можно получить, что в Лагранжиан обобщенные скорости входят только во второй степени, это верно и в случае, когда новые координаты зависят от времени. 
	
	
	Получим уравнения движения. Для этого продифференцируем Лагранжиан по $\dot{q}^p$:
	\[\frac{\partial L}{\partial \dot{q}^p} = \frac{m}{2}\Big( \delta^j_p \dot{q}^k\frac{\partial x^i}{\partial q^j}\frac{\partial x^i}{\partial q^k} + \delta^k_p \dot{q}^j\frac{\partial x^i}{\partial q^j}\frac{\partial x^i}{\partial q^k} \Big) = m\dot{q}^k \frac{\partial x^i}{\partial q^p}\frac{\partial x^i}{\partial q^k} \eqno(15) \]
	Тогда уравнения движения принимают вид:
	\[\frac{d}{dt}\Big( m\dot{q}^k \frac{\partial x^i}{\partial q^p}\frac{\partial x^i}{\partial q^k} \Big) = \frac{\partial }{\partial q^p} \Big( \frac{m}{2}\dot{q}^j\dot{q}^k\frac{\partial x^i}{\partial q^j}\frac{\partial x^i}{\partial q^k} \Big) \eqno(16)\]
	
	\section*{Упражнение 2}
	Пусть дана гладкая замкнутая кривая $\Gamma$ без самопересечений.Необходимо найти кривую, которая при заданной длине ограничивает максимальную площадь.
	
	Для решения введем 2 интегральных представления: площади $A$ и длины $P$, задаваемые следующими формулами:
	\[A[\Gamma] = \int_{S} dxdy = \frac{1}{2}\int_{\Gamma} xdy - ydx = \frac{1}{2} \int_{0}^{1} (x\dot{y} - y\dot{x})dt \eqno(1)\]
	\[P[\Gamma] = \int_{0}^{1} \sqrt{\dot{x}^2+\dot{y}^2} ds \eqno(2)\]
	Если зафиксировать длину кривой, но изменять площадь, то получим задачу на условный экстремум функционала $A[\Gamma]$, причем т.к условие записано в интегральном представлении, то для нахождения нужной кривой достаточно исследовать на безусловный экстремум функционал:
	\[F[\Gamma](\lambda) = A[\Gamma] - \lambda P[\Gamma] = \int_{0}^{1} \Big(\frac{1}{2}(x\dot{y}-y\dot{x}) - \lambda\sqrt{\dot{x}^2+\dot{y}^2}\Big)dt\;\;\;\text{где $\lambda \in \mathbb{R}$} \eqno(3)\]
	Для этого воспользуемся уравнение Эйлера-Лагранжа:
	\[
	\frac{\partial L}{\partial q_i} = \frac{d}{dt}\frac{\partial L}{\partial \dot{q_i}} \Leftrightarrow
	\begin{cases}
	\frac{1}{2}\dot{y} = \frac{d}{dt}\Big( -\frac{y}{2} - \frac{\lambda\dot{x}}{\sqrt{\dot{x}^2 + \dot{y}^2}} \Big) \\
	-\frac{1}{2}\dot{x} = \frac{d}{dt}\Big( \frac{x}{2} - \frac{\lambda\dot{x}}{\sqrt{\dot{x}^2 + \dot{y}^2}} \Big)
	\end{cases} \Leftrightarrow 
	\begin{cases}
	\frac{d}{dt}\Big( y + \frac{\lambda\dot{x}}{\sqrt{\dot{x}^2 + \dot{y}^2}} \Big) = 0 \\
	\frac{d}{dt}\Big( x - \frac{\lambda\dot{x}}{\sqrt{\dot{x}^2 + \dot{y}^2}} \Big) = 0
	\end{cases} \eqno(4)\]
	Из системы следует, что выражения под знаком дифференциала сохраняются, т.е. являются константами:

	
	\[\begin{cases}
	y + \frac{\lambda\dot{x}}{\sqrt{\dot{x}^2 + \dot{y}^2}} = C_1 \\
	x - \frac{\lambda\dot{x}}{\sqrt{\dot{x}^2 + \dot{y}^2}} = C_2		
	\end{cases} \Leftrightarrow 
	\begin{cases}
	C_1 - y = \frac{\lambda\dot{x}}{\sqrt{\dot{x}^2 + \dot{y}^2}}\\
	x - C_2 = \frac{\lambda\dot{x}}{\sqrt{\dot{x}^2 + \dot{y}^2}}
	\end{cases} \Rightarrow (x-C_1)^2+(y-C_2)^2 = \lambda^2 \eqno(5)\]
	  
	Получилось уравнение окружности, причем константу $\lambda$ можно найти из условия для периметра $-2\pi \lambda = P$, тогда все потенциально подходящие кривые описываются уравнением:
	\[\boxed{(x-C_1)^2+(y-C_2)^2 = \frac{P^2}{4\pi^2}} \eqno(6)\]
	Проверим теперь, что найденное решение дает нам максимальную площадь. Для этого рассмотрим новый функционал, который по своей сути является функционалом $F[\Gamma]$, но кривая теперь незамкнута.
	\[J[u](\lambda) = \int_{a}^{b} (u - \lambda \sqrt{1+u'^2} )dx = \int_{a}^{b}\Psi(x,u,u')dx \eqno(7)\]
	\begin{figure}[h!]
		\center{\includegraphics[scale=0.25]{circle.png}}
		\caption{Интегрирование производится по верхней дуге}
	\end{figure}


	Алгоритм исследования функционала на экстремум взят из книги\cite[p.~9]{variations}.
	
	Проверим, что уравнение (6) обеспечивает максимальность функционала (3). Для этого необходимо проверить 2 условия: 
	\begin{enumerate} 
		\item Услове Лежандра --- $\frac{\partial^2 \Psi}{\partial u'^2} >0  \; \forall x \in (a,b)$
		\item Условие Якоби
	\end{enumerate}
	С условием Лежандра все просто. Учет $-2\pi\lambda = P$ дает нам:
	\[\frac{\partial^2 \Psi}{\partial u'^2} = \frac{P}{2\pi\sqrt{(1+u'^2)^3}} > 0 \; \forall u' \in \mathbb{R}  \eqno(8)\]
	Теперь условие Якоби:
	\[\frac{d}{dx}\Big( \frac{\partial^2 \Psi}{\partial u'^2} V' \Big) - \Big(\frac{\partial^2 \Psi}{\partial u^2} - \frac{d}{dx}\frac{\partial^2 \Psi}{\partial u \partial u'}\Big)V = 0 \Leftrightarrow \frac{d}{dx}\Big( \frac{PV'}{2\pi\sqrt{(1+u'^2)^3}}  \Big) = 0 \eqno(9)\]
	\[\frac{V'}{\sqrt{(1+u'^2)^3}} = A_1 \Rightarrow \int_{V_0}^{V(x)}dV = A_1 \int_{a}^{x}\sqrt{(1+u'^2)^3}dx \eqno(10)\]
	Теперь вспомним, что вообще $u(x)$ описывает дугу окружности:
	\[u = C_2 + \sqrt{\lambda^2 - (x-C_1)^2} \Rightarrow u' = -\frac{(x-C_1)}{\sqrt{\lambda^2-(x-C_1)^2}} \eqno(11)\]
	Подставляем в (10):
	\[A_1 \int_{a}^{x}dx\sqrt{\Big(\frac{\lambda^2}{\lambda^2-(x-C_1)^2}\Big)^3} = A_1\lambda^3 \int_{a-C_1}^{x-C_1} (\lambda^2 - x^2)^{-3/2}dx = \Big\{ x = -\lambda\cos(\theta)\Big\} = \]
	\[= A_1\lambda \int_{\arccos(\frac{a-C_1}{\lambda})}^{\arccos(\frac{x-C_1}{\lambda})} sin^{-3}(\theta) \sin(\theta) d\theta = -A_1\lambda \cot(x) \Big |_{\arccos(\frac{a-C_1}{\lambda})}^{\arccos(\frac{x-C_1}{\lambda})} = V(x) - V_0 \Leftrightarrow \]
	\[V(x) = V_0 + \frac{A_1P}{2\pi}\Bigg(\frac{x-C_1}{\sqrt{\lambda^2-(x-C_1)^2}}  - \frac{a-C_1}{\sqrt{\lambda^2-(a-C_1)^2}}\Bigg) \eqno(12)\]
	Пусть начальные условия $V(a) = 0 , V'(a) = 1$, тогда очевидно, что $V_0 = 0$. Разберемся с $A_1$, для этого соотношение (11) подставим в (10) при $x = a$:
	\[u'^2 +1 = \frac{\lambda^2}{\lambda^2 - (x-C_1)^2} \Rightarrow A_1 = \Bigg(\frac{\lambda^2 - (a-C_1)^2}{\lambda^2} \Bigg)^{3/2} > 0 \eqno(13)\]
	Непосредственно из уравнения (10) следует, что функция $V(x)$ строго возрастает, а значит $\forall x \in (a,b) \rightarrow V(x)>0$. Таким образом мы установили, что условие Якоби выполнено, т.е окружность огриничивает наибольшую площадь.
	
	\section*{Задача 1}
	Пусть дан шарик и две точки в пространстве с известными координатами. Какую форму желоба нужно сделать, чтобы шарик без начальной скорости скатился под действием силы тяжести от одной точки к другой за наименьшее время?
	
	
	Для удобства введем систему координат как показано на рисунке:
	\begin{figure}[h!]
		\center{\includegraphics[scale=0.5]{task2_system.jpg}}
	\end{figure}
	Пусть параметризация кривой имеет вид: $y = y(x)$, тогда длина элемента дуги определяется соотношением:
	\[dl = \sqrt{1+y'^2}dx \eqno(1)\]
	Из теоремы об изменении кин энергии находим:
	\[ v = \sqrt{\varkappa y} \eqno(2)\]
	Тогда за время $dt = dl / v$ шарик пройдет кусочек дуги, а значит для оптимизации времени спуска можно ввести функционал:
	\[S[y(x)] = \int_{0}^{l} \frac{\sqrt{1+y'^2}}{\sqrt{\varkappa y}} dx \eqno(2)\]
	Для удобства рассчетов таскать за собой $\sqrt{\varkappa}$ не будем. Обозначим подынтегральное выражение за $L(y,y')$.
	
	Можно и дальше упрощать жизнь: для этого стоит заметить, что $L$ не зависит явно от $x$, а это нам очень помогает, ведь тогда верен переход:
	\[\frac{dL}{dx} = \frac{\partial L}{\partial y} y' + \frac{\partial L}{\partial y'} y'' + \frac{\partial L}{\partial x}\;\;\; \Leftrightarrow\;\;\; \frac{dL}{dx} = \frac{\partial L}{\partial y} y' + \frac{\partial L}{\partial y'} y'' \;\;\; \Leftrightarrow \;\;\; \frac{\partial L}{\partial y'} y' = \frac{dL}{dx} - \frac{\partial L}{\partial y'} y'' \eqno(3)\]
	Для оптимизации интегрального функционала используем уравнение Эйлера-Лагранжа:
	\[\frac{\partial L}{\partial y} = \frac{d}{dx} \frac{\partial L}{\partial y'} \eqno(4)\]
	Домножим обе части на $y'$ и воспользуемся подстановкой из выражения (3):
	\[\frac{dL}{dx} = y''\frac{\partial L}{\partial y'} + y' \frac{d}{dx} \frac{\partial L}{\partial y'} = \frac{d}{dx}\Big( y' \frac{\partial L}{\partial y'}\Big) \;\; \Leftrightarrow \;\; \frac{d}{dx}\Big(L - y' \frac{\partial L}{\partial y'} \Big) = 0 \eqno(5)\]
	Поучаем дифференциальное уравнение:
	\[L - y' \frac{\partial L}{\partial y'} = C \eqno(6)\]
	Подставляем $L$ в уравнение (6):
	\[\frac{\sqrt{1+y'^2}}{\sqrt{y}} - y' \frac{y'}{\sqrt{y}\sqrt{1+y'^2}} = C\;\; \Leftrightarrow \;\; \frac{1}{\sqrt{y}\sqrt{1+y'^2}}=C \;\; \Leftrightarrow \;\; (1+y'^2)y = (1/C)^2 \equiv k \eqno(7)\]
	\[y'^2 = \frac{k}{y}-1 \;\; \Rightarrow \;\; y' = \sqrt{\frac{k-y}{y}} \Leftrightarrow \sqrt{\frac{y}{k-y}} dy = dx \eqno(8)\]
	\[y = k\sin^2 \theta \Rightarrow dy = 2k\sin \theta \cos \theta d\theta  \eqno(9)\]
	\[\int_{0}^{\theta_0} \Big|\frac{\sin\theta}{\cos \theta}\Big| 2k \cos \theta \sin \theta d \theta = k \int_{0}^{\theta_0}(1-\cos(2\theta)) d\theta = \frac{k}{2}\Big(2\theta_0 - \sin(2\theta_0)\Big) = x(\theta_0) \eqno(10)\]
	Получаем кривую, заданную параметрически:
	\[\boxed{
	\begin{cases}
	x(\theta) = \frac{k}{2}\Big(2\theta - \sin(2\theta)\Big) \\
	y(\theta) = k\sin^2(\theta)
	\end{cases} \forall \theta \in \left[0;\frac{\pi}{2}\right]}\eqno(11)\]
	Числа $k,\theta_0$ находится из системы (11) при подстановке $x(\theta_0) = l$ и $y(\theta_0) = H$.
	

	\pagebreak
	\section*{Задача 2}
	Рассмотрим обычный маятник, точка подвеса которого может совершать движение по какому-то произвольному, но известному закону $\textbf{r}(t)$. Сам маятник представляет собой невесомую палку длины $l$ и с массой $m$ на конце.
	\subsection*{1. Лагранжиан системы}
	Для однозначного определения положения маятника нам достаточно знать где находится точка подвеса и 2 угла, задающие точку на сфере, причем скорость маятника определяется соотношением:
	\[\vec{V_A} = \vec{V_0} + \vec{\omega}\times \vec{l} \eqno(1.1)\]
	\begin{figure}[h!]
		\center{\includegraphics[scale=0.5]{pendulum.png}}
	\end{figure}


	Тогда лагранжиан системы примет вид:
	\[L = \frac{m(\vec{V_0}(t) + \vec{\omega}\times \vec{l})^2}{2} + m\vec{g}\Big(\;\vec{r}(t)+\vec{l}\;\Big) \eqno(1.2)\]
	\[L = \frac{m}{2}V_0^2 + \frac{m}{2}\omega^2 l^2 + m(\vec{V_0},\vec{\omega} \times \vec{l}) + m\vec{g}\Big(\;\vec{r}+\vec{l}\;\Big) \eqno(1.3)\]

	\subsection*{2. Конкретный закон $r(t)$}
	Пусть теперь $\vec{r}(t) = r_0 \cos (\gamma t) \cdot (\cos \theta,\sin \theta)$, тогда лагранжиан примет вид:
	\[L = \frac{m}{2} \gamma ^2 r_0^2 \sin^2(\gamma t) + \frac{m}{2}\dot{\phi}^2 l ^2 - m \gamma r_0 \sin(\gamma t) \dot{\phi} l \cos(\phi - \theta) - mg\Big( r_0 \cos(\gamma t) \sin(\theta) - l\cos(\phi) \Big) \eqno(2.1) \]
	
	\begin{figure}[h!]
		\center{\includegraphics[scale=0.5]{pendulum_k.png}}
	\end{figure}


	Выкидываем все полные производные из Лагранжиана:
	\[L = \frac{m}{2}\dot{\phi}^2l^2 -  m \gamma r_0 \sin(\gamma t) \dot{\phi} l \cos(\phi - \theta) +mgl\cos(\phi) \eqno(2.2)\]
	Если ввести обозначения $\tau = \omega_0 t$, $\omega_0 = \sqrt{g/l}$, $A = \frac{\gamma}{\omega_0}\frac{r_0}{l}$, $B = \frac{\gamma}{\omega_0}$ то можно обезразмерить Лагранжиан ($m=1$):
	\[\tilde L = \frac{\dot{\phi}^2}{2\omega_0^2}-A\frac{\dot{\phi}}{\omega_0}\sin(\gamma t)\cos(\phi-\theta) + \cos(\phi) \eqno(2.3)\]
	Параметр $A$ характеризует отношение максимальных скоростей точки подвеса $(\gamma r_0)$ и математического маятника $(\omega_0 l)$. Последним штрихом будет нахождение связи между производными $\phi'_\tau$ и $\phi'_t$:
	\[\frac{d\phi}{dt} = \frac{d\phi}{d\tau} \frac{d\tau}{dt} \Leftrightarrow \frac{\phi'_t}{\omega_0} = \phi'_\tau \eqno(2.4)\] 
	Собираем все вместе:
	\[ \tilde L =  \frac{{\phi'}_{\tau}^2}{2} - A\phi'_\tau\sin(B \tau)\cos(\phi-\theta) + \cos(\phi) \eqno(2.5)\]
	Находим уравнения движения:
	\[ \frac{d}{d\tau}\Big(\phi'_\tau - A\sin(B \tau)\cos(\phi-\theta) \Big) = -A\phi'_\tau\sin(B \tau)\sin(\phi-\theta) - \sin(\phi) \Leftrightarrow \]
	\[ \phi''_\tau + A\phi'_\tau\sin(B \tau)\sin(\phi-\theta) - AB\cos(B\tau)\cos(\phi-\theta) = A\phi'_\tau\sin(B \tau)\sin(\phi-\theta) - \sin(\phi) \Leftrightarrow \]
	\[ \phi''_\tau = AB\cos(B\tau)\cos(\phi-\theta)-\sin(\phi) \eqno(2.6)\]
	Или в старых обозначениях:
	\[\ddot{\phi}+\omega_0^2\sin(\phi) = \frac{r_0}{l}\gamma^2 \cos(\gamma t) \cos(\phi-\theta) \eqno(2.7)\]
	\subsection*{3. Вспомогательная задача}
	
	
	Рассмотрим колебание грузика на пружинке жесткости $k$, к которому приложена быстрая периодическая сила $F = mf_0 \cos(\omega t)$,  $\omega \gg \sqrt{k/m}=\omega_0$.
	
	
	Уравнение движения:
	\[x'' + \omega_0^2 x = f_0 \cos(\omega t)\;\;\;x(0)=0,\;\;\;x'(0)=v_0 \eqno(3.1)\]
	Точное решение данной задачи Коши:
	\[x(t) = \frac{f_0}{\omega^2-\omega_0^2}\cos(\omega_0 t)+\frac{f_0}{\omega_0^2-\omega^2}\cos(\omega t) + \frac{v_0}{\omega_0}\sin(\omega_0 t) \eqno(3.2)\]
	\begin{figure}[h!]
		\center{\includegraphics[scale=0.25]{solution_pendulum.png}}
		\caption{График решения $x(t)$ при $\omega/\omega_0 = 55$   }
	\end{figure}
	
	\pagebreak
	Перепишем уравнение (3.2) и учтем, что $v_0/\omega_0 \gg f_0 / (\omega^2-\omega^2_0)$, при $\omega \gg \omega_0$:
	\[x(t) \approx \frac{f_0}{\omega_0^2-\omega^2}\cos(\omega t) + \frac{v_0}{\omega_0}\sin(\omega_0 t) \eqno(3.3)\]
	Вид уравнения (3.3) позволяет сделать некоторые выводы относительно решения:
	\begin{enumerate} 
		\item В системе есть 2 характерных времени $\tau_1 = 1/\omega_0$ и $\tau_2 = 1/\omega$. Это связано с тем, что решение уравнения состоит из двух гармоник. Гармоника с частотой $\omega_0$ отвечает за движение маятника без учета внешней силы: $x_0(t)$ --- невозмущенное решение. Вторая гармоника задает смещение относительно $x_1(t)$ и появляется из-за внешней силы.
		\item Смещение относительно невозмущенного решения задается формулой:
		\[x_1(t) = \frac{f_0}{\omega_0^2-\omega^2}\cos(\omega t) \eqno(3.4)\]
		При $\omega \gg \omega_0$ можно разложить $x_1(t)$ по степеням $\omega_0 /\omega$. Первый порядок дает нам ответ:
		\[x_1(t) \approx -\frac{f_0}{\omega^2}\cos(\omega t ) \eqno(3.5)\]
	\end{enumerate}

	\subsection*{4. Быстрые и медленные переменные}
	Вернемся к исследованию решения уравнения (2.7). Решение будем искать в виде суммы $\phi(t) = \psi(t)+\chi(t)$, где $\psi(t)$ --- периодическое решение с частотой $\omega_0$, а $\chi(t)$ --- периодическое решение с частотой $\gamma$ и амплитудой $ \ll 1$.
	
	
	Малось амплитуды решения с частотой $\gamma$ следует из соотношения (3.5). Действительно, если подставить в качестве $f_0 = \frac{r_0}{l} \gamma^2 cos(\phi-\theta)$, то амплитуда $\chi$ будет порядка $r_0/l \ll 1$.
	
	Подставим теперь в уравнение (2.7) вместо $\phi(t)$ сумму $\psi(t)+\chi(t)$, раскладывая функции по степеням $\chi$ до первого порядка.
	\[\ddot{\psi}+\ddot{\chi} + \omega_0^2\sin\psi + \chi \omega_0^2 \cos \psi = \frac{r_0}{l}\gamma^2\cos(\gamma t)\Big(\cos(\psi-\theta)-\chi\sin(\psi-\theta)\Big) \eqno(4.1)\]
	Т.к функция $\chi(t)$ осциллирует с частотой $\gamma$, то для получения уравения на $\chi(t)$ необходимо сгруппировать все слагаемые осциллирующие с той же частотой (или достаточно близкой к ней).
	\[\ddot{\chi} + \chi \omega_0^2 \cos \psi = \frac{r_0}{l}\gamma^2\cos(\gamma t)\Big(\cos(\psi-\theta)-\chi\sin(\psi-\theta)\Big) \eqno(4.2)\]
	Теперь вспоминаем, что на самом деле $\chi \ll 1$ и слагаемые содержащие $\chi$ можно выкинуть. Но выкидывать $\ddot{\chi}$ \textbf{НЕЛЬЗЯ} т.к. амплитуда второй производной порядка $\frac{r_0}{l}\gamma \gg \frac{r_0}{l}$.
	\[\ddot{\chi} = \frac{r_0}{l}\gamma^2\cos(\gamma t)\cos(\psi - \theta) \Rightarrow \chi(t) = -\frac{r_0}{l}\cos(\gamma t)\cos(\psi - \theta)\eqno(4.3)\]
	В последнем переходе мы интегрировали вторую производную, считая что $\psi$ на временах порядка $1/\gamma$ практически не меняется.
	
	Усредним уравнение (4.1) по периоду $T = 2\pi/\gamma$, в таком случае все слагаемые содержащие $\chi$ в нечетных степенях обратятся в 0, а величины, которые осциллируют с частотой $\omega_0 \ll \gamma$ практически не изменяются.
	\[\ddot{\psi} + \omega_0^2 \sin\psi  = \frac{r_0^2\gamma^2}{l^2} <\cos^2(\gamma t)>_T \sin(\psi-\theta)\cos(\psi-\theta) = \frac{r_0^2\gamma^2}{2l^2}\sin(\psi-\theta)\cos(\psi-\theta) \eqno(4.4)\]
	
	Преобразуем выражение:
	\[\ddot{\psi} = -\frac{d}{d\psi}\Bigg( \frac{r_0^2\gamma^2}{4l^2}\cos^2(\psi-\theta) - \omega_0^2\cos\psi  \Bigg)\eqno(4.5)\]
 	Выражение в скобках можно заменить функцией $U_{\text{эфф}}(\psi)$.
 	\[U_{\text{эфф}}(\psi) = \frac{r_0^2\gamma^2}{4l^2}\cos^2(\psi-\theta) - \omega_0^2\cos\psi \eqno(4.6)\]
 	
 	\pagebreak
 	\subsection*{5. Положения равновесия и их устойчивость}
 	\subsubsection*{1 случай $\theta = \frac{\pi}{2}$} 
 	Положения равновесия соответствуют экстремумам потенциалной энергии. Используем коэффециент $A$, введенный во 2 пункте, тогда при $\theta = \pi/2$ потенциал можно переписать:
 	\[U = \omega_0^2\Big(\frac{A^2}{4}\sin^2(\psi)-\cos(\psi)\Big) \eqno(5.1.1)\]
 	\[\frac{dU}{d\psi} = \omega_0^2 \Big(\frac{A^2}{2}\sin\psi\cos\psi + \sin\psi \Big) \eqno(5.1.2)\]
 	\[\frac{d^2U}{d\psi^2} = \omega_0^2 \Big(\frac{A^2}{2}\cos(2\psi) + \cos\psi \Big) \eqno(5.1.3)\]
 	Точки экстремума отвечают углам: $0,\;\pi,\;\arccos(-2A^{-2})$, причем положение $\psi = 0$ устойчиво, а $\psi = \arccos(-2A^{-2})$ устойчивым не является:
 	\[\frac{d^2U}{d\psi^2} (\arccos(-2A^{-2})) = \omega_0^2 \Big(\frac{A^2}{2}(2(2A^{-2})^{2}-1) -2A^{-2} \Big) = \omega^2_0\Big(\frac{2}{A^2}-\frac{A^2}{2}\Big) < 0\eqno(5.1.4)\]
 	Данное соотношение верно, если существует $\arccos(-2A^{-2})$ т.е. $2A^{-2}<1 \Leftrightarrow \sqrt{2} < A \Leftrightarrow \sqrt{2gl}<\gamma r_0$.
 	В точке $\psi = \pi$ устойчивость определяется параметром A:
 	\[\frac{d^2U}{d\psi^2} (\pi) = \omega_0^2 \Big(\frac{A^2}{2} - 1 \Big) \eqno(5.1.5)\]
 	При $\sqrt{2} < A \Leftrightarrow \sqrt{2gl}<\gamma r_0$ положение является устойчивым, в противном случае равновесие неустойчиво.
 	
 	\subsubsection*{1 случай $\theta = 0$}
 	 Положения равновесия соответствуют экстремумам потенциалной энергии. Используем коэффециент $A$, введенный во 2 пункте, тогда при $\theta = 0$ потенциал можно переписать:
 	 \[U = \omega_0^2\Big(\frac{A^2}{4}\cos^2(\psi)-\cos(\psi)\Big) \eqno(5.2.1)\]
 	 \[\frac{dU}{d\psi} = \omega_0^2 \Big(-\frac{A^2}{2}\sin\psi\cos\psi + \sin\psi \Big) \eqno(5.2.2)\]
 	 \[\frac{d^2U}{d\psi^2} = \omega_0^2 \Big(-\frac{A^2}{2}\cos(2\psi) + \cos\psi \Big) \eqno(5.2.3)\]
 	Точки экстремума отвечают углам: $0,\;\pi,\;\arccos(2A^{-2})$, причем положение $\psi = 0$ неустойчиво, а $\psi = \arccos(2A^{-2})$ устойчиво:
 	\[\frac{d^2U}{d\psi^2} (\arccos(2A^{-2})) = \omega_0^2 \Big(-\frac{A^2}{2}(2(2A^{-2})^{2}-1) + 2A^{-2} \Big) = \omega^2_0\Big(-\frac{2}{A^2}+\frac{A^2}{2}\Big) > 0\eqno(5.2.4)\]
 	Данное соотношение верно, если существует $\arccos(-2A^{-2})$ т.е. $2A^{-2}<1 \Leftrightarrow \sqrt{2} < A \Leftrightarrow \sqrt{2gl}<\gamma r_0$.
 	В точке $\psi = \pi$ устойчивости нет:
 	\[\frac{d^2U}{d\psi^2} (\pi) = \omega_0^2 \Big(-\frac{A^2}{2} - 1 \Big) < 0 \eqno(5.2.5)\]
 	
 	
 	
	\pagebreak
	\begin{thebibliography}{9} 
		\bibitem{variations}А.В. Ожегова, Р.Г. Насибуллин, Методические указания к решению
		"простейшей задачи" вариационного
		исчисления, Казань, 2013. 
		\bibitem{landau} Ландау Л. Д., Лифшиц Е. М. Механика. — Издание 4-е, исправленное. — Москва: Наука, 1988.
	\end{thebibliography}
	


\end{document}